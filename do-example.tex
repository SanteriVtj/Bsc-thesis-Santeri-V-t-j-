\begin{figure}[h]
        \begin{center}
            \begin{tabular}{c|c}
                \begin{adjustbox}{valign=c}
                    \centering
                    \begin{tikzpicture}
                        \node[circle,draw] (1) at (4,2.5) {$X$};
                        \node[circle,draw] (2) at (0,0) {$Y$};
                        \node[circle,draw] (3) at (4,0) {$V_2$};
                        \node[circle,draw] (4) at (0,2.5) {$V_1$};
                        \node[circle,draw,dashed] (5) at (2, 3.5) {$U_{1}$};
                        \draw[-{Latex[length=2mm]}]  (1) edge (2);
                        \draw[-{Latex[length=2mm]}]  (1) edge (3);
                        \draw[-{Latex[length=2mm]}]  (4) edge (1);
                        \draw[-{Latex[length=2mm]}]  (4) edge (2);
                        \draw[-{Latex[length=2mm]}]  (3) edge (2);
                        \draw[dashed, -{Latex[length=2mm]}]  (5) edge (1);
                        \draw[dashed, -{Latex[length=2mm]}]  (5) edge (4);
                    \end{tikzpicture}
                \end{adjustbox}
                \hspace{1cm}
                &
                \hspace{1cm}
                \begin{adjustbox}{valign=c}
                    \centering
                    \begin{tikzpicture}
                        \node[circle,draw] (1) at (4,2.5) {$X$};
                        \node[circle,draw] (2) at (0,0) {$Y$};
                        \node[circle,draw] (3) at (4,0) {$V_2$};
                        \node[circle,draw] (4) at (0,2.5) {$V_1$};
                        \node[circle,draw,dashed] (5) at (2, 3.5) {$U_{1}$};
                        \node[circle,draw] (6) at (2,1.5) {$\tilde X$};
                        \draw[-{Latex[length=2mm]}]  (6) edge (2);
                        \draw[-{Latex[length=2mm]}]  (6) edge (3);
                        \draw[-{Latex[length=2mm]}]  (4) edge (1);
                        \draw[-{Latex[length=2mm]}]  (4) edge (2);
                        \draw[-{Latex[length=2mm]}]  (3) edge (2);
                        \draw[dashed, -{Latex[length=2mm]}]  (5) edge (1);
                        \draw[dashed, -{Latex[length=2mm]}]  (5) edge (4);
                    \end{tikzpicture}
                \end{adjustbox}
            \end{tabular}
        \end{center}
    \caption{Graphs with two backdoor paths. Graph on left is showing empirical graph on which Pearl's model operates. Right is Heckman's addition to model showing hypothetical model where value of $\tilde X$ is added and edges emerging from $X$ are moved to emerge from that.}
    \label{fig:do example}
\end{figure}